% Some commands used in this file
\newcommand{\package}{\emph}

\chapter*{Introduction}
\addcontentsline{toc}{chapter}{Introduction}

The use of Internet of Things (IoT) is continuously growing, and its application areas henceforth reach all societal and economical fields. IoT networks involve several applications such as smart agriculture, health care, smart homes ... etc. The number of connected IoT devices is increasing exponentially \cite{helpnet}, and so is the amount of their exchanged data. Some of these data might be sensitive or private. To ensure their security and integrity, encryption using Group Key Management (GKM) schemes were proposed. These schemes assume the presence of a centralized Key Manager (KM), which is regarded as the central component of the system.

Previous efforts by researchers affiliated with \emph{Heudiasyc laboratory UMR CNRS 7153} led to the development of the Multi Group Key Management protocol (MGKMP). The protocol already introduces a heap of new features and advantages. But number of its aspects are yet to be studied and improved. A part of the laboratory's work line is to to further push the research and development of MGKMP.

The internship splits into two main stages. The first stage is research oriented. Its mission is part of the laboratory's ongoing research activities on IoT security and GKM. As mentioned, it consists in the study of previous work done by the research team in charge, and elaborate upon it. The task was to study the MGKMP related issues deeper and try to adress them. Hence, the first stage was concluded with the publication of a research article in the proceedings of the \emph{International Conference on Communications Software (SoftCOM 2021)}.

The second stage was more oriented towards parctical engineering. The main objective was to acquire a solid culture in IoT security engineering in practice, and better apprehend the challenges involved.

The content of this report is organized as follows. Chapter~\ref{chap:litterature} reviews and discusses related academic works. Chapter~\ref{chap:mgkmp} studies the architecture of MGKMP and dig into some of its improvment axis. The main task dedicated to work the KM issue of single point of failure out is presented in Chapter~\ref{chap:key_manager}. This includes the problematic definition (Section~\ref{sec:km_problem}), solution design (Section~\ref{sec:election-based-scheme}), simulations and results evaluation (Section~\ref{sec:simulations_results}). Finally, Chapter~\ref{chap:iot_security} sums the second stage up, mainly studying different aspects of IoT security engineering.