\chapter*{Résumé}
%\addcontentsline{toc}{chapter}{Acknowledgement}

Malgré le déploiment à large échelle des réseaux IoT durant les dernières années, la sécurité de ces réseaux deumeurent encore un enjeu majeure. Plusieurs travaux de recherches menés et solutions proposées ont essayé de résoudre ce problème.

Une des solutions courantes repose sur le chiffrement des communications au sein de ces réseaux. Des chercheurs du laboratoire \emph{Heudiasyc laboratory UMR CNRS 7153} ont développé un protocole de gestion des clés cryptographiques pour les réseau IoT multi-groupes (MGKMP). L'objectif principal du stage est de continuer les recherches en cours au sein du laboratoire d'accueil.

La mission primaire du stage est de pousser le travail conduit sur MGKMP. En particulier, il était question d'aborder de plus près les problématiques liées au gestionnaire de clés. Une autre tâche secondaire consistait en l'étude de la sécurité IoT dans l'industrie en général. Les missions du stage ont été globalement accomplies avec un certain niveau de succès. En l'occurance, les travaux ont aboutit à des certains résultats plutôt satisfaisants.

\textbf{Mots clés:} Internet of Things (IoT), Gestion de clés cryptographiques, energie, Cluster Head (CH), sécurité IoT