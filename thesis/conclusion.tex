\chapter*{Conclusion}
\addcontentsline{toc}{chapter}{Conclusion}

The internship took place in two main parts, fundamental research and practical engineering. The first one was a furtherance the laboratory's research work on IoT security in general, and Group Key Management (GKM) in particular. During the second stage, I was rather wearing an engineer's hat. I was still working on IoT security though, but more from a practical view.

The first part was by far longer and much more intense compared to the second one. This part itself went through different steps. The first step of my research task was dedicated to self-learning, documentation and foremost reading academic papers. Although I have a cybersecurity background, I started the internship with almost no prior academic knowledge about security in the IoT domain. I was even much less ranged when it comes to GKM related topics. Therefore, I had a lot to catch up for at first, in order to better apprehend different scientific advances and challenges. I had to study as much as possible the concepts I was going to work on. This concerns different GKM schemes and the MGKMP at first. But then it extended to the Cluster Head (CH) architectures, when I started considering it for the developing solution. 

The second step is an in-depth study of MGKMP. The primary purpose was to identify eventual contribution axes to the ongoing research effort. The forthcoming challenges identified included possible optimizations of the protocol's theoretical design and fixing some vulnerable aspects of the protocol's features. The ultimate goal has always been making a contribution to the existing work in order to produce a research article.

The third and final step of the first stage was to focus on a specific problem. The Key Manager's issue of single point of failure was considered. A solution based on an election scheme was developed. Simulations showed some promising results. A lot of work is yet to be done though, especially for the experimental part.

By the end of the first stage, that ultimate goal mentioned had been fulfilled. A paper summing up the developed solution was published in the proceedings of the \emph{International Conference on Communications Software (SoftCOM 2021)}. This soluion was without a doubt crucial for the system. It solves the targeted problem and does what it is supposed to do. This has already been confirmed by international reviewers from the \emph{SoftCOM 2021} conference. Nevertheless, I still have doubts regarding its utility and practical implementability. The work was carried out under a fundamental research program. But at some point, it has to be transposed into a practical engineering solution. Otherwise, the whole effort would be pointless. That's why I do believe this research should never be disconnected from the field realm. Permanent links and practical evaluations should be conducted all over the way, to make sure that program doesn't loose its main objective from its sight. The program's ultimate purpose is to defend an IoT network against threats. However, how good do we know this threat and how does it operate ? Will this system really deter a malicious hacker from attacking, even a little bit ? A field study of the attackers operational mode is crucial, to avoid defending the front door while the attacker is striking pr sweeping in from behind. Foremost, its important to be confident that we are not switching to new security systems, which look safer from the outside, whereas it's rotten to the core from the inside. This said, it doesn't in any way systematically implies the system is useless. Because so much of the work achieved is brilliant and academics have confirmed it. But it's only to point some deficiency out, regarding the overall approach in the problem resolution.

I gained a lot of experience in this stage. It involved a tremendous professional research know-how. This includes how to conduct a bibliographic work, how to read academic papers and criticize them, how to manage articles using academic software (\emph{Zotero} in my case) and write my own article in \LaTeX. The research requires a lot of precision, thinking and synthesis. One always has to take into account other researchers work and their outcome. The solution development require permanent focus on the ultimate goal, in a way that despite all the sideline details, one never gets diverted or distracted. The writing of the article and its publication gave me a closer insight of the academic domain.

Second part was focused on the IoT security industry. It consists mainly in studying common IoT security risks and industrial solutions from a practical point of view. Working on real IoT systems and malware samples was at most beneficial, as I gained useful knowledge and first hand experience on IoT engineering. It raised my awareness about several types of threats, and made me look closer at different engineering solutions. I didn't go far concerning the proprietary solutions bit though, since it looked like most of them are basically snake oil commercial products.

I found the internship's overall unfolding very satisfactory on the personal level. I worked on a research task, and thus I have now a better aptitude to work in the research field. I also had some engineering tasks, and thus I gained knowledge in case I want to work in the industry. This internship actually helped me a lot in my personal development and my professional plans. Besides, the pedagogic objective of an internship is to enhance my operability after graduating. This objective couldn't have been better fulfilled.